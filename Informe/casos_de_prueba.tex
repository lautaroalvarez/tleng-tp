\section{Casos de prueba}

\subsection{Objectos con arrays anidados}
Es un caso raro porque al haber arrays anidados se dan casos con guiones repetidos en una misma linea, indentación que aumenta y disminuye y objetos dentro de arrays.

\textbf{Entrada:}
\begin{verbatim}
{
    "a": [
        1,
        {
            "b": 2,
            "c": "asd",
            "d": []
        },
        [
            [3],
            4,
            [
                5,
                6,
                [
                    [7]
                ]
            ]
        ],
        8
    ]
}
\end{verbatim}

\textbf{Salida del programa:}
\begin{verbatim}
a:
- 1
- b: 2
  c: "asd"
  d: []
- - - 3
  - 4
  - - 5
    - 6
    - - - 7
- 8
\end{verbatim}

\subsection{Objetos con arrays de objetos}
Este caso estaba bueno para ver cómo se comportaba el programa con objectos, mezclados con arrays y objetos anidados. Al haber un objeto en una posición de un array la indentación de la segunda clave tiene que ser igual que la primera, pero sin guión.

\textbf{Entrada:}
\begin{verbatim}
{
  "a" :1,
  "c":[1,2,3],
  "b":[3,4,{ "n":[2,3,3,{"o":{"p":"fgh"}}],"m":"a"},3],
  "d": {
    "a": 1,
    "b": "asd"
  }
}
\end{verbatim}

\textbf{Salida del programa:}
\begin{verbatim}
a: 1
c:
- 1
- 2
- 3
b:
- 3
- 4
- n:
  - 2
  - 3
  - 3
  - o:
      p: "fgh"
  m: "a"
- 3
d:
  a: 1
  b: "asd"
\end{verbatim}

\subsection{Array directo}
Algo interesante de este caso es que es directamente un array, y no un objeto como veníamos probando. Además se incluye un objeto y un array vacíos. También pusimos todo en la misma linea y sin espacios.

\textbf{Entrada:}
\begin{verbatim}
["asd",0,[],{"a":{},"b":1},[1,2]]
\end{verbatim}

\textbf{Salida del programa:}
\begin{verbatim}
- "asd"
- 0
- []
- a: {}
  b: 1
- - 1
  - 2
\end{verbatim}

\subsection{Clave sin comillas}
En este caso ingresamos un objeto con una clave que no está entre comillas. Al no ser un string válido devuelve error.

\textbf{Entrada:}
\begin{verbatim}
{a:1}
\end{verbatim}

\textbf{Salida del programa:}
\begin{verbatim}
Illegal character 'a'
\end{verbatim}

\subsection{Array mal cerrado}
En este caso ingresamos un json con una llave faltante.

\textbf{Entrada:}
\begin{verbatim}
{"a":[1,2,[]}
\end{verbatim}

\textbf{Salida del programa:}
\begin{verbatim}
Error de sintaxis!
\end{verbatim}

\subsection{Elemento de objeto vacío}
En este caso ingresamos un json con un elemento vacío.

\textbf{Entrada:}
\begin{verbatim}
{"a":"una prueba", ,}
\end{verbatim}

\textbf{Salida del programa:}
\begin{verbatim}
Error de sintaxis!
\end{verbatim}
